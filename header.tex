% =========================================================================== %
% Generic LaTeX template for a document
% Copyright (C) 2015  Federico "MrModd" Cosentino (http://mrmodd.it/)
% Original template by Ted Pavlic (http://www.tedpavlic.com) and downloaded
% from http://www.latextemplates.com
%
% This program is free software: you can redistribute it and/or modify
% it under the terms of the GNU General Public License as published by
% the Free Software Foundation, either version 3 of the License, or
% (at your option) any later version.
%
% This program is distributed in the hope that it will be useful,
% but WITHOUT ANY WARRANTY; without even the implied warranty of
% MERCHANTABILITY or FITNESS FOR A PARTICULAR PURPOSE.  See the
% GNU General Public License for more details.
% =========================================================================== %

\usepackage{fancyhdr} % Required for custom headers
\usepackage{extramarks} % Required for headers and footers
\usepackage{courier} % Required for the courier font


% --------------------------------------------------------------------------------------
% DOCUMENT DEFINITIONS: MODIFY THESE LINES
% --------------------------------------------------------------------------------------

% If you'd like to delete one of the following tags comment the
% relative line on front.tex file
\newcommand{\authorsname}{Author One, Author Two} % Editor list
\newcommand{\firstlinetitle}{This is the Title}
\newcommand{\secondlinetitle}{Second line Title}
\newcommand{\thirdlinetitle}{Something else}
\newcommand{\subtitle}{I'm a subtitle!!} % Tagline(s) or further description
\newcommand{\documentdate}{\today} % Publishing date (you can also define a custom date)
%\newcommand{\documentdate}{23/05/2014} % <--For example this way

% --------------------------------------------------------------------------------------



% Margins
\topmargin=-0.45in
\evensidemargin=0in
\oddsidemargin=0in
\textwidth=6.5in
\textheight=9.0in
\headsep=0.25in

\linespread{1.1} % Line spacing

% Set up the header and footer
\pagestyle{fancy}
\lhead{\authorsname} % Top left header
\chead{\firstlinetitle} % Top center head
\rhead{\leftmark} % Top right header
\lfoot{} % Bottom left footer
\cfoot{\documentdate} % Bottom center footer
%\rfoot{Page\ \thepage\ of\ \protect\pageref{LastPage}} % Bottom right footer
% Bottom right footer is redefined in index.tex to differentiate roman and arabian numerations
\renewcommand\headrulewidth{0.4pt} % Size of the header rule
\renewcommand\footrulewidth{0.4pt} % Size of the footer rule

% Spacing between paragraphs
\setlength{\parskip}{\baselineskip} % Indent and leave some vertical space
%\setlength\parindent{0pt} % No indent or vertical space
% Comment both lines for just indent and no vertical space

%----------------------------------------------------------------------------------------
%	DOCUMENT STRUCTURE COMMANDS
%	Skip this unless you know what you're doing
%----------------------------------------------------------------------------------------

% Header and footer for when a page split occurs within a problem environment
\newcommand{\enterProblemHeader}[1]{
\nobreak\extramarks{#1}{#1 continued on next page\ldots}\nobreak
\nobreak\extramarks{#1 (continued)}{#1 continued on next page\ldots}\nobreak
}

% Header and footer for when a page split occurs between problem environments
\newcommand{\exitProblemHeader}[1]{
\nobreak\extramarks{#1 (continued)}{#1 continued on next page\ldots}\nobreak
\nobreak\extramarks{#1}{}\nobreak
}

%\setcounter{secnumdepth}{0} % Uncomment this if you want to remove default section numbers
\newcounter{homeworkProblemCounter} % Creates a counter to keep track of the number of problems

\newcommand{\homeworkProblemName}{}
\newenvironment{homeworkProblem}[1][Problem \arabic{homeworkProblemCounter}]{ % Makes a new environment called homeworkProblem which takes 1 argument (custom name) but the default is "Problem #"
\stepcounter{homeworkProblemCounter} % Increase counter for number of problems
\renewcommand{\homeworkProblemName}{#1} % Assign \homeworkProblemName the name of the problem
\section{\homeworkProblemName} % Make a section in the document with the custom problem count
\enterProblemHeader{\homeworkProblemName} % Header and footer within the environment
}{
\exitProblemHeader{\homeworkProblemName} % Header and footer after the environment
}

\newcommand{\problemAnswer}[1]{ % Defines the problem answer command with the content as the only argument
\noindent\framebox[\columnwidth][c]{\begin{minipage}{0.98\columnwidth}#1\end{minipage}} % Makes the box around the problem answer and puts the content inside
}

\newcommand{\homeworkSectionName}{}
\newenvironment{homeworkSection}[1]{ % New environment for sections within homework problems, takes 1 argument - the name of the section
\renewcommand{\homeworkSectionName}{#1} % Assign \homeworkSectionName to the name of the section from the environment argument
\subsection{\homeworkSectionName} % Make a subsection with the custom name of the subsection
\enterProblemHeader{\homeworkProblemName\ [\homeworkSectionName]} % Header and footer within the environment
}{
\enterProblemHeader{\homeworkProblemName} % Header and footer after the environment
}
